\chapter{Conclusion}

It is without a fraction of a doubt that the project can be considered a tremendous success. Initial feedback from the customer has been immensely positive, in particular thanks in part to being able to take criticism and requests from a meeting and reproducing them in code for the very next week. Combining a unique approach in the operations view, along with a more traditional calendar and PDF exportation capability was absolutely the correct approach and allows for an immense quantity of flexibility in how a user interacts with the software.

If it were possible to start the project from scratch, more time would be dedicated to the initial resarch into the domain area and to more thoroughly research and discuss keywords and terminology where there are any doubts or areas of interpretation, to ensure that the customer and developer remain on the same page at all times. In addition, taking the additional effort to adhere to a strict agile methodology, such as Extreme Programming (XP) would likely have proved a better solution than making adaptations and producing a hybrid solution, as having a concrete methodology to fall back on when issues arise would have helped better ensure that timelines are adhered to, and that development remains disciplined at all times - ensuring that work is marked as 'done' and tests are produced to prohibit regressions without having to go back and produce tests at a later stage.

Furthermore, despite it being a strength of the project, more communication with the customer would have been a fantastic addition to the workflow, in particular setting up live demos or sending demo versions of the software for users to play with and provide feedback on - to better find what works, and what does not.

If it were possible to continue the project for another month, as previously stated, improving code quality and maintainability would be the first priority.

If the developer were in charge of their own grading, it would be considered at least high Merit level because:

\begin{itemize}
    \item The software works tremendously well and fulfills 80\% of required functionality, is easy to use, and fulfills the criteria to be a replacement for the customer's current spreadsheet based system
    \item Whilst unique and hybridised, a great deal of effort was put into researching appropriate agile methodologies, and adhering to it.
    \item A wealth of research was carried out into alternatives existing in this already niche field, whilst touching on an important and historic computational problem in the Nurse Scheduling Problem. (NST)
    \item This project is ready for the real-world and can be used from day \#1 by Corris Railway, which was one of the motivations for undertaking the project in the first place.
    \item An immense amount of learning has been facilitated from this project undertaking - having learned a great deal about agile software development, the Laravel framework, PHP, and indeed, steam railways and how they operate.
\end{itemize}

In conclusion, this project has been an immense pleasure to implement and has as aforementioned, provided an unparalled learning experience that has reflected much that has been learned throughout the year. The end result is a CRUD-compliant, full-stack web application with a full testing suite that uses modern web technologies and conforms to PSR PHP web standards, ready for a life time of success and continued development in the hands of the Corris Railway, located in beautiful Machynlleth.